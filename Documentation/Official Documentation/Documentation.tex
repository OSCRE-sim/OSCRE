\documentclass[12pt]{article}
\usepackage{amsmath,amssymb,amsfonts}
\usepackage{graphicx}
\usepackage{hyperref}
\usepackage{listings}
\usepackage{geometry}
\usepackage{fancyhdr}
\geometry{margin=1in}
\hypersetup{
    colorlinks=true,
    linkcolor=blue,
    filecolor=magenta,
    urlcolor=cyan,
}
\pagestyle{fancy}
\fancyhf{}
\fancyhead[L]{Comprehensive Project Documentation}
\fancyhead[R]{\thepage}

\title{Comprehensive Project Documentation: Open-Source Radiation Hardening Simulator}
\author{Jacob Anderson, David Nichols, Collin Lambert, Parker Allred}
\date{\today}

\begin{document}

\maketitle
\tableofcontents
\newpage

\section{Project Overview}
This project aims to develop an open-source radiation hardening simulator using Xschem and NGSpice. The simulator provides a comprehensive library for simulating the effects of radiation on electronic circuits, including modules for fault injection and radiation effect simulation. The project is designed to be user-friendly and accessible to researchers and engineers working in the field of radiation hardening.

\section{Library Structure and Core Features}
\subsection{Single Event Effect Simulation}
Single event effects are radiation effects that are induced by a single radiation strike event. Such Events include, single event transients, single event upsets, and many more. The following three modules are used to simulate single event transients and by extrapolation, single event upsets.

To use any of the following modules, if simulating a radiation strike on an NMOS, place the module such that the input of the current source is connected to the drain node, and the output of the current source is connected to the body node. 

For a PMOS, place the module such that the input of the current source is connected to the body node, and the output of the current source is connected to the drain node.

\subsubsection{Double Exponential}
The double exponential current source has been a standard method of simulating the effects of a single radiation event for many years. To use, one must specify the rise time, fall time, rise time constant, fall time constant, and total amount of charge to inject.
\subsubsection{Dual Double Exponential}
Similar to the double exponential current source, the dual double exponential current source utilizes two double exponential current sources added together. This has been shown to be a more accurate representation of the actual photocurrents generated in a radiation strike. All the parameters specified in the double exponential are also present in the dual double exponential but there are two sets of them that must be specified.
\subsubsection{Adaptive Double Exponential}
The adpative double exponential current source addresses some issues that arise from the independent double exponential and dual double exponential current sources. These independed models don't take into account the circuitry surrounding the component on which a radiation strike is being simulated. As such, unrealistic values can be generated by these independed sources. To solve this, the adpative double exponential current source has circuitry that allows it to take into account the effects that surrounding circuitry create. This prevents unrealistic values from being generated. 
\subsection{Other Radiation Effect Simulation}
\subsubsection{Total Ionizing Dose}
\subsubsection{Rail Span Collapse}


\section{Installation Instructions}
To install the simulator, follow these steps:

\subsection{Prerequisites}
Before installing the simulator, ensure you have the following software installed:
\begin{itemize}
    \item Homebrew (for macOS users)
    \item Git
\end{itemize}

\subsection{Running the Installation Script}
1. Clone the project repository from GitHub:
    \begin{lstlisting}[language=bash, breaklines=true]
    git clone https://github.com/Jacoba1100254352/RAD-HARD.git
    cd RAD-HARD
    \end{lstlisting}

2. Run the installation script:
    \begin{lstlisting}[language=bash, breaklines=true]
    ./install_script.sh
    \end{lstlisting}

The script will automatically install all necessary dependencies, including xschem, NGSpice, Tcl, Tk, and BeSpiceWave.app, and set up the environment for you.

\section{Usage Examples}
\subsection{Example Circuit: Memory Cell}
\begin{figure}[htbp]
%\centering
%\includegraphics[width=0.8\linewidth]{example_circuit_memory.png}
%\caption{Example Memory Cell Circuit}
%\label{fig:memory_cell}
\end{figure}

\textbf{Steps to Simulate:}
\begin{enumerate}
    \item Create the schematic in xschem.
    \item Export the netlist as \texttt{memory\_cell.spice} by pressing the "netlist" button at the top right
    \item configure xschem to run the simulation for you by going to 
	\begin{itemize}
	\item Simulation\textgreater Configure simulators and tools
	\item go to the "spice" section and select the option that says: ngspice -b -r "$n.raw" "$N"
	\item press the "simulate" button at the top right of xschem
\end{itemize}	    
	\item View waveforms by pressing "waves\textgreater external viewer" in the top right of xschem or manually opeing the file in BeSpiceWave
\end{enumerate}

\subsection{Example Circuit: Operational Amplifier}
\begin{figure}[htbp]
%\centering
%\includegraphics[width=0.8\linewidth]{example_circuit_opamp.png}
%\caption{Example Operational Amplifier Circuit}
%\label{fig:opamp}
\end{figure}

\textbf{Steps to Simulate:}
\begin{enumerate}
    \item Create the schematic in xschem.
    \item Export the netlist as \texttt{opamp.spice} by pressing the "netlist" button at the top right
    \item Alternatively, configure xschem to run the simulation for you by going to 
	\begin{itemize}
	\item Simulation\textgreater Configure simulators and tools
	\item go to the "spice" section and select the option that says: ngspice -b -r "$n.raw" "$N"
	\item press the "simulate" button at the top right of xschem
\end{itemize}	    
	\item View waveforms by pressing "waves\textgreater external viewer" in the top right of xschem or manually opeing the file in BeSpiceWave
\end{enumerate}

\section{Additional Resources}
\subsection{Documentation and Tutorials}
\begin{itemize}
    \item User Guide: Detailed user guide with step-by-step instructions.
    \item API Documentation: Comprehensive API documentation for all modules and functions.
    \item Tutorials: Various tutorials to help users get started with the simulator.
\end{itemize}

\subsection{GitHub Repository}
The source code and additional resources for the project can be found on GitHub:
\begin{itemize}
    \item \href{https://github.com/Jacoba1100254352/RAD-HARD}{GitHub Repository: RAD-HARD}
\end{itemize}

\section{Conclusion}
This document provides comprehensive documentation for the open-source radiation hardening simulator project. By following the instructions and utilizing the provided resources, users can effectively simulate and analyze the effects of radiation on electronic circuits.

\section*{Acknowledgment}
We would like to thank Dr. Shiuh-hua Wood Chiang for his guidance and support throughout this project. This work was supported by [Funding Source].

\end{document}