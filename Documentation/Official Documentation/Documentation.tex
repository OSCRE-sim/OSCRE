\documentclass[12pt]{article}
\usepackage{amsmath,amssymb,amsfonts}
\usepackage{graphicx}
\usepackage{hyperref}
\usepackage{listings}
\usepackage{geometry}
\usepackage{fancyhdr}
\geometry{margin=1in}
\hypersetup{
    colorlinks=true,
    linkcolor=blue,
    filecolor=magenta,
    urlcolor=cyan,
}
\pagestyle{fancy}
\fancyhf{}
\fancyhead[L]{Comprehensive Project Documentation}
\fancyhead[R]{\thepage}

\title{Comprehensive Project Documentation: Open-Source Radiation Hardening Simulator}
\author{Jacob Anderson, David Nichols, Collin Lambert, Parker Allred}
\date{\today}

\begin{document}

\maketitle
\tableofcontents
\newpage

\section{Project Overview}
This project aims to develop an open-source radiation hardening simulator using xschem and NGSpice. The simulator provides a comprehensive library for simulating the effects of radiation on electronic circuits, including modules for fault injection, radiation effect simulation, and results analysis. The project is designed to be user-friendly and accessible to researchers and engineers working in the field of radiation hardening.

\section{Library Structure and Core Features}
\subsection{Fault Injection Module}
The Fault Injection Module simulates faults in circuit elements to study radiation effects.

\textbf{Key Functions:}
\begin{itemize}
    \item \texttt{DefineFaultModel(type, parameters)}: Defines the fault model.
    \item \texttt{InjectFault(circuit)}: Injects faults into the circuit.
    \item \texttt{LogFault(details)}: Logs the fault details.
\end{itemize}

\subsection{Radiation Effect Simulation Module}
The Radiation Effect Simulation Module simulates the effects of radiation on electronic circuits.

\textbf{Key Functions:}
\begin{itemize}
    \item \texttt{SimulateSET(circuit)}: Simulates Single Event Transients.
    \item \texttt{SimulateSEU(circuit)}: Simulates Single Event Upsets.
    \item \texttt{AnalyzeImpact(data)}: Analyzes the impact of radiation on the circuit.
\end{itemize}

\subsection{Results Analysis Module}
The Results Analysis Module analyzes and presents simulation results.

\textbf{Key Functions:}
\begin{itemize}
    \item \texttt{GenerateReport(results)}: Generates a report of the simulation results.
    \item \texttt{PlotResults(data)}: Plots the results for visualization.
    \item \texttt{ComputeSER(data)}: Computes the Soft Error Rate.
\end{itemize}

\section{Installation Instructions}
\subsection{Prerequisites}
Before installing the simulator, ensure you have the following software installed:
\begin{itemize}
    \item Homebrew (for macOS users)
    \item Git
    \item xschem
    \item NGSpice
    \item GTK+3
\end{itemize}

\subsection{Step-by-Step Installation}
\begin{enumerate}
    \item Install Homebrew (macOS):
    \begin{lstlisting}[language=bash, breaklines=true]
    /bin/bash -c "$(curl -fsSL https://raw.githubusercontent.com/Homebrew/install/HEAD/install.sh)"
    \end{lstlisting}
    
    \item Install Dependencies:
    \begin{lstlisting}[language=bash, breaklines=true]
    brew install gtk+3 cairo pango autoconf automake libtool pkg-config at-spi2-core
    \end{lstlisting}

    \item Clone the xschem-gaw Repository:
    \begin{lstlisting}[language=bash, breaklines=true]
    git clone https://github.com/StefanSchippers/xschem-gaw.git
    cd xschem-gaw
    \end{lstlisting}

    \item Generate Configuration Files and Build:
    \begin{lstlisting}[language=bash, breaklines=true]
    autoreconf --install
    automake --add-missing
    ./configure
    make
    sudo make install
    \end{lstlisting}

    \item Add the following to your \texttt{.bashrc} or \texttt{.zshrc} file:
    \begin{lstlisting}
    export NO_AT_BRIDGE=1
    \end{lstlisting}

    \item Source the \texttt{.bashrc} or \texttt{.zshrc} file or open a new terminal window:
    \begin{lstlisting}[language=bash]
    source ~/.bashrc  # or
    source ~/.zshrc
    \end{lstlisting}

    \item Run the GTK Analog Wave Viewer:
    \begin{lstlisting}[language=bash]
    gaw
    \end{lstlisting}
\end{enumerate}

\section{Usage Examples}
\subsection{Example Circuit: Memory Cell}
\begin{figure}[htbp]
%\centering
%\includegraphics[width=0.8\linewidth]{example_circuit_memory.png}
%\caption{Example Memory Cell Circuit}
%\label{fig:memory_cell}
\end{figure}

\textbf{Steps to Simulate:}
\begin{enumerate}
    \item Create the schematic in xschem.
    \item Export the netlist as \texttt{memory\_cell.spice}.
    \item Run the simulation with NGSpice:
    \begin{lstlisting}[language=bash, breaklines=true]
    ngspice -b memory_cell.spice -o ngspice_output.txt
    \end{lstlisting}
    \item Analyze the results using the Results Analysis Module.
\end{enumerate}

\subsection{Example Circuit: Operational Amplifier}
\begin{figure}[htbp]
%\centering
%\includegraphics[width=0.8\linewidth]{example_circuit_opamp.png}
%\caption{Example Operational Amplifier Circuit}
%\label{fig:opamp}
\end{figure}

\textbf{Steps to Simulate:}
\begin{enumerate}
    \item Create the schematic in xschem.
    \item Export the netlist as \texttt{opamp.spice}.
    \item Run the simulation with NGSpice:
    \begin{lstlisting}[language=bash, breaklines=true]
    ngspice -b opamp.spice -o ngspice_output.txt
    \end{lstlisting}
    \item Analyze the results using the Results Analysis Module.
\end{enumerate}

\section{Additional Resources}
\subsection{Documentation and Tutorials}
\begin{itemize}
    \item User Guide: Detailed user guide with step-by-step instructions.
    \item API Documentation: Comprehensive API documentation for all modules and functions.
    \item Tutorials: Various tutorials to help users get started with the simulator.
\end{itemize}

\subsection{GitHub Repository}
The source code and additional resources for the project can be found on GitHub:
\begin{itemize}
    \item \href{https://github.com/Jacoba1100254352/RAD-HARD}{GitHub Repository: RAD-HARD}
\end{itemize}

\section{Conclusion}
This document provides comprehensive documentation for the open-source radiation hardening simulator project. By following the instructions and utilizing the provided resources, users can effectively simulate and analyze the effects of radiation on electronic circuits.

\section*{Acknowledgment}
We would like to thank Dr. Shiuh-hua Wood Chiang for his guidance and support throughout this project. This work was supported by [Funding Source].

\end{document}